
\documentclass[14pt]{beamer}

\usepackage{amsmath}
\usepackage{mathrsfs}
\usepackage{graphicx}

\usepackage[english]{babel}

%\AtBeginDvi{\special{pdf:tounicode 90ms-RKSJ-UCS2}} 
%\AtBeginDvi{\special{pdf:tounicode EUC-UCS2}}

\usetheme{Boadilla}
\usecolortheme{default}
\usefonttheme{professionalfonts}

\setbeamertemplate{navigation symbols}{}
\setbeamertemplate{blocks}[rounded][shadow=false]
%\setbeamercolor{alartblock}{fg=red}

\usebackgroundtemplate

\usepackage{setspace}

\title[Master's Thesis]{How Does Financial Incentive to Hospital Affect Inpatient Care?} 
%\title[Master's Thesis]{How Does Financial Incentive to Hospital Affect Physician's Medical Decision?} 
%\subtitle{Evidence from Nonlinear Reimbursement in Japan}
\subtitle{Evidence from Reimbursement System in Japan}
\author[Masaki Takahashi]{Masaki Takahashi}
\institute[Univ. of Tokyo]{Graduate School of Economics, University of Tokyo}
\date{January 28, 2015}

\begin{document}

\frame{\titlepage}

\frame{\tableofcontents}

\section{Introduction}

\begin{frame}[allowframebreaks]
\frametitle{Motivation}
\begin{itemize}
\item Under rising medical expenditure, reimbursement becomes more important as financial incentive to contain health care cost.
\item In Japan, per-diem fixed payment system ``DPC" was introduced to reduce unnecessary health care.
\item It is critical to understand how medical provider respond to financial incentive.
%\item It is important to construct reimbursement system that reduce unnecessary health care.
%\item In the case of inpatient care, financial incentive is provided through reimbursement to hospitals.
%\item How physicians respond to financial incentives for hospitals is crucial to health care policy.
%\item In aging society, like Japan, it is important to understand how financial incentives affect medical providers for constructing ``necessary and sufficient" health care system. 

\item Exploiting DPC implementation, we can examine the impact of financial incentive to medical provider mainly through length of stay.
%\item Implementation of DPC provide opportunity to examine the important questions in health care policy.
\end{itemize}

\framebreak
 
Research Questions:
%\vspace{\baselineskip}
\begin{enumerate}
\item Does DPC implementation reduce length of stay? 
%\item If physician has several ways to make a profit for hospital, which ones to be chosen?
\begin{itemize}
\item Is DPC effective to solve long-term hospitalization problem in Japan?
%\item Does hospital flexibly adjust length of stay right after policy change?
\item Compared with other treatment choices?
%\item Reduce length of stay? (Bed management)
%\item Cut cost of input per day? 
%\item Increase provision of profitable treatment? (Supplier-induced demand)
\end{itemize}
\vspace{\baselineskip}
\item How does nonlinear-pricing schedule affect distribution of length of stay?
\begin{itemize}
\item Does hospital discharge patients right before reimbursement drop?
%\item Does distribution of length of stay bunch at discontinuity point of nonlinear per-diem schedule?
%\item What kind of patients are affected by nonlinear-pricing incentive?
\end{itemize}
\end{enumerate}
\end{frame}

\begin{frame}
\frametitle{What I do}
\begin{enumerate}
\item Estimate the impact of DPC implementation on length of stay and other variables in DID framework using patient level data.
\vspace{\baselineskip}
\item Estimate the increase in discharged patients right before reimbursement drop using method proposed by Chetty et al.(2011).
%\item Estimate the bunching in distribution of length of stay at the discontinuity point of nonlinear schedule using method proposed by Chetty et al.(2011 QJE).
\end{enumerate}
\end{frame}

\begin{frame}
\frametitle{Main Fidings}
First question:
\begin{itemize}
\item Length of stay is reduced by 0.85 days due to DPC implementation.
\item Treatment choice is less affected by DPC than length of stay is.
%\item Profitable treatment (operation in this case) were NOT increased.
\end{itemize}
Second question:
\begin{itemize}
\item The number of discharged patients increases right before reimbursement drop.
\item Nonlinear incentive is concentrated on patients under short hospitalization.
%\item Nonlinear incentive is more significant for patients whose reimbursent is composed largely of nonlinear schedule.
%\item Distribution of length of stay bunch at kink of nonlinear schedule.
%\item Bunching is more significant for patients whose reimbursent is composed largely of nonlinear incentive.
%\item Distribution does not bunch at kink point of DPC group-specific schedule.
\end{itemize}
\end{frame}

\section{Institutional Background}

\begin{frame}[allowframebreaks]
\frametitle{Institutional Background}
Traditionally, reimbursement had been paid through fee-for-service (FFS) in Japan.\par
\vspace{5pt}
\centering{
{\small Figure 1. FFS Payment System}\par
\includegraphics[width=6.5cm]{ffsfigure.png}
}

%\begin{itemize}
%\item Hospitals (and physicians) have little incentive to cut cost because of cost-plus-margin payment.
%\item Hospital basic charge were paid by per-diem schedule, and the payment is dropped at 14th and 30th day of hospitalization.
%\end{itemize}

\framebreak

\begin{center}
{\small Figure 2. Per-diem Hospital Basic Charge of FFS}
\includegraphics[width=10.5cm]{schedule_ffs2.png}
\end{center}

\framebreak

%\begin{itemize}
\flushleft{
In 2003, alternative payment system called DPC was introduced for inpatient care.
}
\vspace{5pt}

\centering{
{\small Figure 3. DPC Payment System}\par
\includegraphics[width=7cm]{dpcfigure.png}
}

%\item Medical procedures regarding operation are paid through FFS as before.
%\item Other procedures such as medication, injection, diagnostic imaging, are paid through per-diem fixed payment regardless of cost of input.
%\item Schedule of per-diem payment differs among DPC groups that are divided based on disease and method of treatment.
%\end{itemize}

\framebreak

\begin{center}
{\small Figure 4. Per-diem Fixed Payment of DPC}
\includegraphics[width=10.5cm]{schedule_dpc3.png}
\end{center}
\end{frame}

\section{Financial Incentives}

\begin{frame}[allowframebreaks]
\frametitle{Financial Incentives}
\begin{itemize}
\item DPC reduces length of stay if hospital improve bed turnover rate to keep profitable short-term hospitalization.
\item If hospitals cannot admit sufficient new patients, length of stay might not be reduced.
%DPC implementation gave hospitals incentives to discharge patients under long-term hospitalization and admit new patients.
%\vspace{\baselineskip}
\item Additionally, DPC possibly induce hospitals to;
\begin{itemize}
\item Increase frequency and/or input of surgery.
\item Reduce medical input in ward.
\end{itemize}
\end{itemize}

\framebreak

Length of stay has important role in nonlinear-pricing schedule.
\begin{itemize}
\item Under nonlinear-pricing, distribution of length of stay is expected to ``bunch" at kink point\par (see Figure 5).
\item By changing from FFS to DPC, kinks at 14th and 30th day of hospitalization are eliminated.
\end{itemize}
$\Rightarrow$ Bunching at kinks should also be eliminated.

\framebreak

\begin{center}
{\small Figure 5. Bunching at Kink}
\includegraphics[width=8cm]{bunching.png}
\end{center}

\end{frame}

\section{Data}

\begin{frame}[allowframebreaks]
\frametitle{Data}
\begin{itemize}
\item Use medical records of circulatory disease patients in DPC database in 2008 and 2009.
%\item Adoption of DPC is decided by each hospital and the trend of adoption peaked at this period.
\item Each hospital can decide when (whether) to adopt DPC.
\item Before adoption, hospitals submit medical records under FFS payment for two years.
\item Our sample contains only hospitals which are willing to adopt DPC.
%\item We can also observe medical records of preparation period.

\end{itemize}

\framebreak

Exploiting one year difference in timing of adoption.

\begin{center}
{\small Figure 6. Variation in payment system}
\includegraphics[width=10cm]{timing.png}
\end{center}

%\framebreak

%Difference in the timing of adoption generates variation of payment system among hospitals.



\framebreak

\begin{center}
\includegraphics[width=9cm]{summarystat.png}
\end{center}

\end{frame}

\section{Estimation}

\begin{frame}[allowframebreaks]
\frametitle{Regression Estimation}
Basic specification:
\begin{equation}
%\begin{split}
LoS_{iht} = t_{\it 2009} + \alpha_{h} + \beta(DPC*t_{\it 2009}) + \delta_{d} + X'_{iht}\gamma + \epsilon_{iht}
%\end{split}
\end{equation}
$t_{\it 2009}=1$ if $t=2009$\par
$DPC=1$ if hospital is in treatament group\par
$\alpha_{h}$\ \ : Hospital dummy (336 hospitals).\par
$\delta_{d}$\ \ \ : DPC group dummy (150 groups).\par
$X_{iht}$: Patient's characteristics.\par
$\epsilon_{iht}$\  : Error term (clustered at hospital level).\par
\vspace{\baselineskip}
Estimate with negative binomial regression.

\framebreak

\begin{itemize}
\item Following are also used as dependent variable:
  \begin{itemize}
  \item Surgery dummy
  \item ln(Input of Surgery)
  \item ln(Average input in ward per day)
  \end{itemize}
\item Input was proxied by FFS-equivalent reimbursement.
\item In the case of surgery dummy, probit was used and DPC group dummies were dropped.
\item OLS was used for other dependent variables.
\end{itemize}

\end{frame}

\begin{frame}%[allowframebreaks]
\frametitle{Results of Regression Estimation}

\begin{center}
\includegraphics[width=12cm]{results3.png}
\end{center}

\end{frame}

\begin{frame}[allowframebreaks]
\frametitle{Bunching Estimation}
\begin{itemize}
\item Recall that price drop at 15th and 31th day of hospitalization were eliminated by DPC.
\item What happen to distribution of length of stay around kink?
\end{itemize}

\begin{columns}
  \begin{column}{0.5 \textwidth}
    \begin{center}
    {\tiny Figure 2. Per-diem Hospital Basic Charge of FFS}
    \includegraphics[width=6cm]{schedule_ffs2.png}
    \end{center}
  \end{column}

  \begin{column}{0.5 \textwidth}
    \begin{center}
    {\tiny Figure 4. Fixed payment of DPC}
    \includegraphics[width=6cm]{schedule_dpc3.png}
    \end{center}
  \end{column}
\end{columns}
    
\framebreak

Basic idea of Chetty et al. (2011):
\begin{enumerate}
\item Construct ``smooth" counterfactual distribution by polynomial.
\item Estimate how much more patients were discharged at kink than counterfactual one.
\end{enumerate}

\centering{
%{\small Figure 5. Bunching in Distribution}\par
\includegraphics[width=6cm]{exbunch.png}
}

\framebreak

\flushleft{
First, estimate following regression:
\begin{equation}
C_j=\sum_{i=0}^p \beta_{i}\cdot(Z_j)^i + \gamma_i \cdot {\bf 1}[Z_j=k]+\epsilon_j
\end{equation}

$C_j$\ :\ Number of patients discharged at $j$th day.\par
$Z_j$\ \ :\ $j$th day of hospitalization relative to kink.\par
$k$\ \ :\ Kink point.\par
$p$\ \ :\ Degree of polynomial.\par

($p=9$ in basic specification.)
}

\framebreak

Counterfactual distribution is a predicted value of each $C_j$ excluding contribution of kink:

\begin{equation}
{\hat C_j}=\sum_{i=0}^p {\hat \beta_{i}}\cdot(Z_j)^i
\end{equation}

Then, bunching estimate is:

\begin{equation}
{\hat b}=\frac{C_k-{\hat C_k}}{{\hat C_k}}.
\end{equation}

To generate standard error, bootstrap was conducted.
%where $C_k-{\hat C_k}$ is the number of excess discharged patients at kink.

\framebreak

%\begin{itemize}
%\item Bootstrap is conducted to generate standard error.
%\vspace{\baselineskip}
Divide hospitals into following three groups and compare the change of bunching estimates:
\begin{itemize}
\item Hospitals that change from FFS to DPC.
\item Hospitals that keep DPC.
\item Hospitals that keep FFS.
\end{itemize}
%\end{itemize}

\end{frame}

\begin{frame}[allowframebreaks]
\frametitle{Results of Bunching Estimation}
Limit sample to patients without surgery.\par
\vspace{5pt}
\centering{
{\small Figure 7. \par ${\sf FFS (Price\ Drop\ at\ 15th\ Day)} \rightarrow {\sf DPC (No\ Price\ Drop)}$}
\includegraphics[width=12cm]{changehosp_notope_new2.png}
}
\flushleft{
{\scriptsize
Blue line: Observed distribution \par
Red line: Counterfactual distribution
}
}

\framebreak

\begin{center}
{\small Figure 8. ${\sf FFS(Price\ Drop\ at\ 15th\ Day)} \rightarrow {\sf FFS(Price\ Drop\ at\ 15th\ Day)}$}
\includegraphics[width=12cm]{alwaysPREP_notope_new.png}
\end{center}

\framebreak

\begin{center}
{\small Figure 9.\par ${\sf DPC(No\ Price\ Drop)} \rightarrow {\sf DPC(No\ Price\ Drop)}$}
\includegraphics[width=12cm]{alwaysDPC_notope_new.png}
\end{center}

\framebreak

\begin{itemize}
\item Bunching estimate decreases only when reimbursement drop is eliminated.
\item That is, hospital has incentive to discharge patients right before reimbursement drop.
\item Excess discharge patients are reduced by 66.3\%.
\end{itemize}

\end{frame}

\section{Conclusion}

\begin{frame}
\frametitle{Conclusion}
\begin{itemize}
\item There is still room to reduce length of stay in Japan.
\item It is relatively hard to change actual treatment choice through financial incentive.
\item Nonlinear-incentive affect hospital decision on less-serious patients.
%\item Length of stay is reduced significantly by DPC implementation.
%\item Treatment choice is less affected by DPC than length of stay.
%\item Nonliear-incentive is concentrated on less serious patients.
%\item If hospitals basic charge is big part of reimbursement (i.e. patients without surgery), bunching incentive gets stronger.
\end{itemize}

\end{frame}

\begin{frame}[allowframebreaks]
\frametitle{References}
\footnotesize
\begin{spacing}{0.9}
[1] Bajari, Patrick Han Hong, Minjung Park, and Robert Town. 2014. ``Estimating price sensitivity of economic agents using discontinuity in nonlinear contracts." {\it Working Paper}.\par
[2] Chetty, Raj, John N. Friedman, Tore Olsen, and Luigi Pistaferri. 2011. ``Adjustment cost, firm responses, and micro vs macro labor supply elasticities: evidence from danish tax records." {\it Quarterly Journal of Economics}, 126, 749-804.\par
[3] Cutler, David. 1995. ``The incidence of adverse medical outcome under prospective payment." {\it Econometrica}, 63, 29-50.\par
[4] Dafny, Leemore. 2005. ``How do hospitals respond to price changes?" {\it American Economic Review}, 95(5), 1525-1547.\par
[5] Einav, Liran, Amy Finkelstein, and Paul Schrimpf. 2013. ``The response of drug expenditure to non-linear contract design: Evidence from Medicare Part D." {\it Working Paper}.\par
[6] Ellis, Randall and Thomas McGuire. 1996. ``Hospital response to prospective payment: Moral hazard, selection, and practice-style effects." {\it Journal of Health Economics},15(3), 257-277.\par

\framebreak

[7] Ito, Koichiro. 2014. `` Do consumer respond to marginal or average price?: Evidence from nonlinear electricity pricing." {\it American Economic Review}, 104(2), 537-563.\par
[8] Marsh, Christina. 2013. ``Estimating demand elasticities using nonlinear pricing." {\it Working Paper}.\par
[9] Nawata, Kazumitsu and Koichi Kawabuchi. 2010. ``Analysis of length of stay for cataract surgery before and after introduction of Diagnosis Procedure Combination-based inclusive payment system." {\it Japanese Journal of Health Economic and Policy}, 21(3), 291-302.\par
[10] OECD. 2009. ``OECD Economic Survey JAPAN." {\it OECD}.\par
[11]Saez, Emmanuel. 2010. ``Do taxpayers bunch at kink points?" {\it American Economic Journal: Economic Policy}, 2(3), 180-212.\par
[12] Shigeoka, Hitoshi  and Kiyohide Fushimi. 2014. ``Supplier-induced demand for newborn treatment: Evidence from Japan." {\it Journal of Health Economics}, 35, 162-178.\par
[13] Wang, Kai, Ping Li, Ling Chen, Ken Kato, Makoto Kobayashi, and Kazunobu Yamauchi. 2010. ``Impact of the Japanese Diagnosis Procedure Combination-based Payment System in Japan." {\it Journal of Medical Systems},34, 95-100.
\end{spacing}
\end{frame}


\end{document}
















%%%%%%%%%       Keep for a rainy day       %%%%%%%%%

 \framebreak

\begin{block}{Assumption.  (Ellis and McGuire (1986 JHE))}
Physician is an agent for both hospital and patient. %but imperfect for patient in the sense of giving too much weight to hospital's financial benefit.
\end{block}

\begin{itemize}
\item It is unclear which principals are dominant.
\item In Japan, almost all physicians are hired by hospitals for inpatient care (not necessarily in US).\par
\end{itemize}
$\Rightarrow$ It is reasonable that physicians change medical decisions to make a profit for hospitals under policy change. 

\framebreak

\begin{itemize}
\item Bunching at 14th day decreases as kink was eliminated.
\item In the case of all patients, bunching decreases even when kink was NOT eliminated.
\item If we focus on patients without operation, bunching decreases sharply only when kink was eliminated as expected.
\end{itemize}
Note: Bunching does not exist at kink of DPC group-specific kinks.


\framebreak

\begin{itemize}
\item Hospitals flexibly manipulate length of stay to make a profit.
\item Average medical input per day is also reduced but its significant level is larger than length of stay is.
\item Operation is not affected by DPC implementation.
%\item Reduction of the cost of operation is a strong counter evidence of supplier-induced demand.
\end{itemize}

{\small Figure 7. Change of Bunching at Kink (${\sf FFS} \rightarrow {\sf DPC}$, All Patients)}
\includegraphics[width=12cm]{changehosp_new.png}
\end{center}


\framebreak

\begin{center}
{\small Figure 8. ${\sf DPC} \rightarrow {\sf DPC}$, All Patients}
\includegraphics[width=12cm]{alwaysDPC_new.png}
\end{center}

\framebreak

\begin{center}
{\small Figure 9. ${\sf FFS} \rightarrow {\sf FFS}$, All Patients}
\includegraphics[width=12cm]{alwaysPREP_new.png}
\end{center}

\framebreak


